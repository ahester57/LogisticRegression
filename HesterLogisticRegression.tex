
% CS 4710 HW 5

\documentclass{article}
\usepackage{titling}
\usepackage{siunitx}	% for scientific notation
\usepackage{graphicx}
\usepackage{caption}
\usepackage{animate}
\usepackage{listings}
\usepackage{microtype}

\setlength{\droptitle}{-9em}
\emergencystretch=1em
\oddsidemargin=12pt
\textwidth=450pt
\begin{document}

\title{CS 4340 - Logistic Regression}
\author{Austin Hester}
\date{November 5, 2017}
\maketitle

% This makes it so sections aren't automatically numbered
\makeatletter
\def\@seccntformat#1{%
	 \expandafter\ifx\csname c@#1\endcsname\c@section\else
	  \csname the#1\endcsname\quad
  \fi}
\makeatother


%%%%%%%%%%%%%%%%%%%%%%%%%%%%%%%%%%

\section*{Introduction}

We will use logistic regression to obtain probabilities of passing a course given a number of weeks inactive. \\

Our training data is:  

\begin{center}
\begin{tabular}{|c|c|}
	\hline
	\multicolumn{2}{|c|}{\textbf{Training Data}} \\\hline
	\textbf{Weeks Inactive} & \textbf{Pass/Fail} \\\hline
	1 & 0 \\
	2 & 1\\
	3 & 0 \\
	4 & 1\\
	5 & 0\\
	6 & 1\\
	7 & 1\\
	8 & 1\\
	\hline
\end{tabular}

0 = pass, 1 =  fail
\end{center}



%%%%%%%%%%%%%%%%%%%%%%%%%%%%%%%%%%
\newpage
\section*{The Code}
\lstset{
	frame=tb,
	tabsize=4,
	showstringspaces=false
}
\begin{lstlisting}[language=Python,breaklines=true]
# Austin Hester
# Logistic Regression
# CS 4340 - Intro to Machine Learning
# 11.05.17

import numpy as np
# define x_0
x0 = 1

# ln L = sum_1_n{ x_i^j * ( y^j - ( e^{w0x0+w1x1} / (1 + e^{w0x0+w1x1} )))}

# compute d/dwi ln L with given x, y, weights, and step size
def ddw(i, x, y, w_, c):
    s = 0
    # for d/dw1 ln L
    for j in range(1,9):
        
        eexp = np.exp( ( w_[0] * x0 ) + ( w_[1] * x[j-1] ) ) 
        if (i == 0):
            pointmult = 1
        else:
            pointmult = x[j-1]
        point = pointmult * (y[j-1] - ( eexp / ( 1 + eexp ) ) ) 
        s = s + point
    w = w_[i] + (c * s)
    return w

# compute the passing chance given x weeks of inactivity
def passingchance(w_, x):
    chance = 1 / ( 1 + np.exp(x0 * w_[0] + (x * w_[1])))
    return chance

# input data [1-8], "weeks of inactivity"
#x = np.arange(1, 9)
x = np.array( [1,2,3,4,5,6,7,8] )
# output, 0 = "pass", 1 = "fail"
y = np.array( [0,1,0,1,0,1,1,1] )
# step size
c = 0.01
# initial weight vector
w_ = np.array( [1., 1.] )

# iterate T times
T = 2000
for t in range(T):
    new_w0 = ddw(0,x,y,w_,c)
    new_w1 = ddw(1,x,y,w_,c)
    w_[0] = new_w0
    w_[1] = new_w1

# print weight vector
print("\nWeight vector: ", w_)
print("\nWeeks of Inactivity\tChances of passing")
for i in range(0,13):
    print("\t",i, "\t\t", round(passingchance(w_, i),4)*100, "%")

\end{lstlisting}

%%%%%%%%%%%%%%%%%%%%%%%%%%%%%%%%%%

\section*{Notes}

\textbf{(a)} Logistic regression results 
\begin{lstlisting}[breaklines=true,basicstyle=\small]

Weight vector:  [-1.8142984  0.5594476]

Weeks of Inactivity		Chances of passing
	 0 					85.99 %
	 1 					77.81 %
	 2 		 			66.72 %
	 3 		 			53.39 %
	 4 		 			39.57 %
	 5 		 			27.23 %
	 6 		 			17.62 %
	 7 		 			10.89 %
	 8 		 			6.53 %
	 9 		 			3.84 %
	 10 			 	2.23 %
	 11 		 		1.29 %
	 12 		 		0.74 %
\end{lstlisting}

Running logistic regression over our input data gives us a weight vector of $ < -1.81,  0.56 > $. \\

At 3 weeks of inactivity, a student has a 53.4\% chance of passing the course.

At 5 weeks of inactivity, a student has a 27.2\% chance of passing the course. \\

\newpage

\textbf{(b)} Logistic regression can very well be used for classification. \\
We can classify using: \\

if [ $P( Y=0 | X) > P( Y=1 | X)$ ] then pass

\hspace{1em} else fail \\

\textbf{(c)} 



%%%%%%%%%%%%%%%%%%%%%%%%%%%%%%%%%%
\end{document}














